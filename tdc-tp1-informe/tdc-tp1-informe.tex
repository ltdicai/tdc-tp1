
\documentclass[a4paper,spanish]{article}
\usepackage[a4paper,margin=1.5cm,top=0.5cm,bottom=0.5cm]{geometry}
\usepackage[pdftex]{graphicx}

% Paquetes varios, entre otras cosas, para poder escribir con acentos!

\usepackage[T1]{fontenc}
\usepackage{anysize} 
\usepackage[utf8]{inputenc}
\usepackage[spanish]{babel}
\usepackage{hyperref}
\usepackage[pdftex]{graphicx}
\usepackage{amsfonts}
\usepackage{amsmath}
\usepackage{amssymb}
\usepackage{array} 
\usepackage{tabularx}
\usepackage{textcomp}
\usepackage{pdfpages}
\usepackage[small]{caption}

%\usepackage{newclude}

\usepackage{fancyhdr}

\usepackage{paquetes/caratula}		% Si no tienen este paquete en sus paquetes de latex, 

%\hoffset=-2.5cm
%\textwidth=17cm
%\parskip=1ex

\pagestyle{fancy}

% Paquete para poder escribir los simbolos de naturales, enteros, reales...
% Si no lo tienen, hagan --> sudo apt-get install texlive-fonts-extra
% La cagada es que pesa unos 100 MB...

% \usepackage{dsfont}

% Paquetes para escribir algoritmos:

\usepackage{paquetes/algorithm}		% Si no tienen este paquete en sus paquetes de latex, 
                                  % este archivo tiene que estar en la carpeta donde estan nuestros .tex
\usepackage{paquetes/algorithmic}	% Idem que el de arriba

%\input{spanishAlgorithmic}	% Este SI o SI tiene que estar en la carpeta donde estan nuestros  .tex

%\usepackage{moreverb}
%\usepackage{verbatim} %entorno verbatim para codigo fuente

% Aca pueden ver bien como se usa el paquete...
% http://www.rosapolis.net/2008/04/21/escribir-algoritmos-en-latex/

\DeclareGraphicsExtensions{.bmp,.png,.pdf,.jpg,.eps}
\hypersetup {colorlinks=false, pdfborder={0 0 0}}
\newcommand{\fullref}[1]{sección \ref{#1}}
\newcommand{\quotel}{\textquotedblleft}
\newcommand{\quoter}{\textquotedblright \ }

% Cosas del enunciado de ellos

\parskip = 11pt
\newcommand{\real}{\hbox{\bf R}}

% No deja sangria al comienzo de los parrafos!
%\setlength\parindent{0pt}

\begin{document}

\materia{Teoría de las Comunicaciones}
\submateria{}
\titulo{TP1: Wiretapping}
\subtitulo{}
%\grupo{Nombre Grupo}
\integrante{Benitti, Raúl}{592/08}{raulbenitti@gmail.com}
\integrante{Castro, Damián}{326/11}{ltdicai@gmail.com}
\integrante{Lizana, Helen}{118/08}{hsle.22@gmail.com}
\integrante{Grenier, Michelle}{418/10}{michelle.grenier@hotmail.com}

\fecha{15 de julio de 2016}

\maketitle

\tableofcontents
\newpage

%\input{abstract.tex}
%\newpage

%\input{jerarquia.tex}
%\newpage

\section{Introducción}

Las redes de computadoras han dejado de ser 
una tecnología reservada a ciertos ámbitos científicos y militares para convertirse
en piezas fundamentales en el desarrollo de casi cualquier actividad, a tal grado que las relaciones humanas, 
desde el comercio hasta las guerras, han sido profundamente transformadas por la conectividad
alcanzada en los últimos años.  
Es por esto que analizar los distintos aspectos de una red
puede proveer información útil para comprender el uso que se le está dando a la red, información que sirve tanto para
modificar la infraestructura y los protocolos
utilizados a fin de mejorar la calidad del servicio como, incluso, manipular las actividades que se esten llevando sobre ella.


En el presente trabajo experimentaremos sobre sistemas basados en dos de las tecnologías de redes más difundidas, Ethernet (802.3) y WiFi (802.11), y analizaremos los datos obtenidos utilizando dos modelos de fuente de información para extraer conclusiones sobre el uso y la configuración de las redes. Los conceptos teóricos sobre los que basaremos el análisis se presentan a continuación.


\subsection{Información y Fuente de información}

Una fuente de información es todo aquello que emite mensajes de acuerdo a una ley de probabilidad fija. Los mensajes pertenecen a un conjunto finito de símbolos $S={s_{1},...,s_{n}}$, conocido como el alfabeto de la fuente. La emisión de un símbolo $s_i$ por parte de la fuente $S$ representa un evento que tiene asociada una probabilidad fija $P_S(s_i)$ de ocurrir. 

Dado un evento $e$ con probabilidad $P(e)$, se define la \textbf{información del evento $e$} como

\begin{center}
$I(e)=-\log{P(e)}$ 
\end{center}

$I(e)$ es una medida de la cantidad de información que obtenemos por la ocurrencia de E: mientras más improbable sea E, mayor será la información brindada por su ocurrencia (menor será la incertidumbre sobre el hecho observado). Dicho de otra manera, si sabemos que un evento E tiene alta probabilidad de ocurrir, entonces su ocurrencia no aportará mucha información sobre lo que se está observando.



\subsection{Entropía}

Dada una fuente de información $S={s_{1},...,s_{n}}$, se define la entropía de S, $H(S)$, como la suma ponderada de la información de cada símbolo de $S$

$$H(S)=\sum^{n}_{i=1}{P(s_{i})*I(s_{i})}$$

La entropía de una fuente de información mide la cantidad de información esperada al observar la emisión de un nuevo símbolo por parte de la fuente. 

\subsection{ARP}

Para poder realizar envío de paquetes de capa de red utilizando los servicios de capa de enlace es necesario poder realizar un mapeo entre las direcciones de ambas capas. ARP (\textsl{Address Resolution Protocol}) es un protocolo de control que surge como respuesta a esta necesidad. Cada host y switch de una red mantiene una tabla donde se relaciona una dirección lógica $d$ con la direccion física $f$ a la que debe entregarse cualquier paquete destinado a $d$ (el host con direccion física $f$ no es necesariamente el destinatario de la dirección $d$: puede ser un intermediario que sabe cómo hacer llegar el paquete a $d$). En el caso de redes IP sobre Ethernet, el protocolo ARP es utilizado para mapear direcciones IP con direcciones MAC.
La configuración de estas tablas ARP se realiza dinámicamente siguiendo un protocolo que consiste básicamente en los siguientes pasos:

\begin{enumerate}
	\item Un host \textbf{A} desea enviar un paquete a una determinada IP. Si \textbf{A} conoce la dirección MAC a la que debe enviar los paquetes destinados a esa IP, entonces utiliza esa dirección física. Si no, envía un mensaje $broadcast$, o sea a todos los hosts dentro de la red, y aguarda la respuesta. Este mensaje se conoce como \textbf{ARP request} (WHO\_HAS), y lleva la siguiente información:
		\subitem IP origen: IP de A
		\subitem IP destino: IP a la que se desea enviar un paquete
		\subitem MAC origen: MAC de A
		\subitem MAC destino: dirección broadcast de Ethernet (FF:FF:FF:FF)
	\item Si dentro de la red existe un host \textbf{B} que sabe cómo direccionar a la dirección IP requerida, entonces responde al mensaje ARP request con un mensaje \textbf{ARP reply} (IS\_AT) indicando su dirección física MAC. Este host puede ser el dueño de la dirección IP, o un host intermediario (como un router). Además, extrae las direcciones IP origen y MAC origen del paquete ARP request, y actualiza su tabla ARP para relacionarlas.
	El paquete ARP reply contiene la siguiente información:
		\subitem IP origen: IP de B
		\subitem IP destino: IP de A
		\subitem MAC origen: MAC de B
		\subitem MAC destino: MAC de A
	\item \textbf{A} recibe el ARP reply de \textbf{B}, actualiza su tabla ARP y envía el paquete original utilizando la dirección física de \textbf{B}.
\end{enumerate}
Además, cada entrada de las tablas ARP tiene seteado un tiempo de vida. Una vez agotado ese tiempo, la entrada se descarta y debe volver a aprenderse. 



\newpage

\section{Experimentos}

Como mencionamos anteriormente, el análisis de paquetes de una red puede utilizarse para inferir información sobre la actividad y topología de la red. En este trabajo aprovecharemos esta capacidad para dilucidarqué protocolos se distinguen del resto, cuál es la incidencia de los paquetes ARP y cuáles son los nodos destacados de las redes.
Realizamos cuatro experimentos para obtener datos, uno sobre cada una de las siguientes redes:
\begin{itemize}
	\item Red1: Red wiFi de un laboratorio del DC
	\item Red2: Red Wifi de un bar Starbucks
	\item Red3: Red Ethernet en un ámbito laboral
	\item Red4: Red Ethernet en un ámbito laboral
\end{itemize}


Modelamos estas redes como dos fuentes de información distintas:
\begin{enumerate}
	\item $S$: este modelo fue dado por la cátedra. El alfabeto se define como los protocolos enviados dentro de los paquetes Ethernet capturados durante el experimento. Así mismo, consideramos como función de probabilidad a la frecuencia de cada símbolo dentro del experimento, donde marcamos como ocurrencia de un evento a la observación de un protocolo al capturar un paquete.
	\item $S1$: con este modelo deseamos poder distinguir los nodos relevantes de una red dada. Para ello, definimos el alfabeto de $S1$ como las direcciones MAC de los paquetes del protocolo ARP. En este caso también tomamos como función de probabilidad a la frecuencia de cada dirección MAC dentro del total observado, pero contabilizando las ocurrencias de cada MAC segun se muestra en el Cuadro \ref{ARP}.
	
Hemos de aclarar algunos puntos sobre $S1$. La decisión de utilizar direcciones MAC en lugar de direcciones IP radica en el hecho de querer identificar los nodos físicos dentro de la red observada. Utilizar directamente direcciones IP podría llevar a malinterpretar la topología de la red: por ejemplo, podríamos considerar como relevantes a varios host con distintas IP que se encuentran fuera del sistema en estudio, y no notar que todo el táfico debe pasar por un único nodo propio.

Por otra parte, decidimos utilizar las direcciones MAC tal como se muestra en el Cuadro \ref{ARP} pués consideramos que brindan la mayor información acerca de la actividad de cada nodo en la red: enviar un paquete ARP nos da información sobre la existencia del host, y de recibir un paquete ARP (es decir, la existencia de un paquete IS\_AT destinado a un nodo particular) podemos deducir que el nodo destino seguramente continue con envío de más paquetes (es decir, tenga actividad inmediata).

	
		\begin{table}
			\centering
		\begin{tabular}{l l}
				Observación & Eventos contabilizados \\
				\hline
				Paquete WHO\_HAS, MAC origen & 1 evento \\
				Paquete WHO\_HAS, MAC destino & 0 evento \\
				Paquete IS\_AT, MAC origen & 1 evento\\
				Paquete IS\_AT, MAC destino & 1 evento\\
		\end{tabular}
		\caption{Contabilización de eventos para $S1$}
		\label{ARP} 
		\end{table}
\end{enumerate}
asdasd

\subsection{Herramientas de sniffing}

Para capturar y procesar la información, utilizamos tanto el programa \textit{Wireshark} como un script (capturar.py), escrito en Python, utilizando la librería para análisis de redes \textit{scapy}. Ambas herramientas hacen uso del modo promíscuo de la placa de red, en el cual se capturan no solo los paquetes dirigidos a el host que esta capturando, sino todos los paquetes que se envíen por el medio.

En su forma de ejecución básica, el script que desarrollamos muestra por pantalla cada paquete que captura hasta que sea detenido con una interrupción (CTRl+C). Al finalizar, se muestra el total de paquetes capturados, los protocolos de dichos paquetes (junto con la cantidad correspondiente), y la entropía del mode

 la cantidad de paquetes  Incorporamos otros modos de ejecución.se realiza a través del comando siguiente comando (para ver otras opciónes , ejecutar el comando con la opción \textit{-h}.)

\begin{verbatim}
	sudo python capturar.py -i <interfaz_de_captura>
\end{verbatim}



\subsection{Análisis de entropía de una red}
Uno de los ejercicios solicitaba calcular la entropía de una fuente.
Para ello debemos definir con presicion dos cosas, la \textbf{fuente de informacion} y el \textbf{evento}, para luego calcular su probabilidad
y de allí la entropía.

\subsubsection{Experimento 1:}

Definimos los siguientes modelos de fuente de información en un red. Previamente contamos con información de cómo está compuesta dicha red: 2 Notebooks, 2 Smartphones, 1 SmartTV, 1 Desktop/Media Server y 1 Router.

Calcularemos las entropías modificando la \textit{tool} construida. 

\textbf{Modelo 1:}
\begin{itemize}
\item \textbf{Fuente de información:} Considero un símbolo como una dirección IP que emite un paquete ARP del tipo \textbf{is-at}.
\item \textbf{Probabilidad de evento:} Probabilidad de que cierta IP aparezca como emisora de un paquete \textbf{is-at}.
\end{itemize}



\textbf{Modelo 2:}
\begin{itemize}
\item \textbf{Fuente de información:} Considero un símbolo como una dirección IP por la que se consulta en un paquete ARP del tipo \textbf{who-has}.
\item \textbf{Probabilidad de evento:} Probabilidad de que cierta IP aparezca como destino de un paquete \textbf{who-has}.
\end{itemize}

\newpage

\section{Resultados}
En esta sección mostraremos los resultados de los experimentos.

\subsection{Experimento 1:}

Las entropías calculadas para las 2 fuentes de información propuestas fueron:

\begin{center}
\begin{tabular}{ l r }
   Fuente& Entropía  \\
\hline
   1 & 1.371641 \\
   2 & 0.570425 \\
 \end{tabular}
\captionof{table}{entropias}
\label{entropiashogar}
\end{center}

Probabilidades para el modelo 1 según IPs:
%\begin{center}
%\begin{tabular}{ l c r }
%   Fuente& IP &probabilidad \\
%\hline
%   1 & 10.0.0.3  &0.864341085271 \\
%   2 & 10.0.0.2& 0.135658914729\\
%\caption{modelo 1}
% \end{tabular}
%\captionof{table}{modelo 1}
%\end{center}

\begin{center}
\begin{tabular}{ l r }
   IP &Probabilidad \\
\hline
192.168.1.138	& 0.3125 \\
192.168.1.103	& 0.0625 \\
192.168.1.137	& 0.1875 \\
192.168.1.1		& 0.3125 \\
192.168.1.143	& 0.125 \\
 \end{tabular}
\captionof{table}{Modelo 1}
\label{probabilidadesModel1}
\end{center}

Probabilidades para el modelo 2 según IPs:
\begin{center}
\begin{tabular}{ l r }
   IP &Probabilidad \\
\hline
192.168.1.1		& 0.9017 \\
192.168.1.103	& 0.0173 \\
192.168.1.143 	& 0.0115 \\
192.168.1.137	& 0.0173 \\
192.168.1.136	& 0.0057 \\
192.168.1.138	& 0.0462 \\
 \end{tabular}
\captionof{table}{Modelo 2}
\label{probabilidadesModel2}
\end{center}

\begin{figure}[H]
	\centering
	\includegraphics[scale = 0.8]{graficos/emiso_is_at_hogar.pdf}
	\caption{Histograma Prob./IPs Modelo 1}
	\label{histogramaprobabilidadesModel1}
\end{figure}


\begin{figure}[H]
	\centering
	\includegraphics[scale = 0.8]{graficos/recep_who_has_hogar.pdf}
	\caption{Histograma Prob./IPs Modelo 2}
	\label{histogramaprobabilidadesModel2}
\end{figure}

\newpage
\subsection{Experimento 2:}

\begin{center}
\begin{tabular}{ l r }
   Fuente& Entropía  \\
\hline
   1 & 3.533684 \\
   2 & 4.352357 \\
 \end{tabular}
\captionof{table}{entropias experimento 2}
\label{entropiasexperimento2}
 %\end{tabular}
\end{center}

En el caso del primer modelo había muchas IP's con la misma probabilidad y datos no significativos que tuvimos que eliminar,
debido a que perdía mucha claridad el gráfico. Pusimos las IP's con probabilidades más significativas.

\begin{figure}[H]
	\centering
	\includegraphics[scale = 0.6]{graficos/emiso_is_at_empresa.pdf}
	\caption{Histograma Prob./IPs Modelo 1}
	\label{histogramaprobabilidadesModel1}
\end{figure}

Con una \textbf{entropía} de \textbf{3.533684}.


En nuestro segundo experimento realizado en una red empresarial (clase B), una breve recolección de datos para el Modelo 2 (receptores de paquetes who-has) arrojó la siguiente información:

\begin{figure}[H]
	\centering
	\includegraphics[scale = 0.6]{graficos/recep_who_has_empresa.pdf}
	\caption{Histograma Prob./IPs Modelo 1}
	\label{histogramaprobabilidadesatos}
\end{figure}

Con una \textbf{entropía} de \textbf{4.352357}.

\section{Envio de paquetes}
En esta sección vamos a ver graficamente la cantidad de paquetes que se envian entre si los nodos de la red,con la esperanza de poder 
sacar algunas conclusiones. Los pesos en las aristas indican la cantidad de paquetes enviados de una IP a otra.

\subsection{Experimento 1}


\begin{figure}[H]
	\centering
	\includegraphics[scale = 0.6]{graficos/emisoras_is_at_red_hogar.png}
	\caption{paquetes is at emitidos en una red hogareña}
      \label{emisorasisat}
\end{figure}




\begin{figure}[H]
	\centering
	\includegraphics[scale = 0.5]{graficos/receptoras_who_has_red_hogar.png}
	\caption{paquetes who has en una red hogareña}
	\label{emisoraswhohas}	
\end{figure}

\newpage


\subsection{Experimento 2}
En este caso como se trataba de una red empresarial tuvimos que tomar recortes del grafo de envío de paquetes, ya que de lo contrario
se tornaba ilegible. Capturamos solo partes que colaboran con el objetivo de recolectar nodos distinguidos. 
Aclaración: En los 2 siguientes gráficos, el nodo apuntado representa a la IP origen mientras que el nodo apuntador es la IP destino.


\begin{figure}[H]
	\centering
	\includegraphics[scale = 0.5]{graficos/receptores_who_has_parte1_atos.png}
	\caption{paquetes who has en una empresarial}
	\label{receptoraswhohasatos}	
\end{figure}


\begin{figure}[H]
	\centering
	\includegraphics[scale = 0.5]{graficos/receptores_who_has_atos_parte2.png}
	\caption{paquetes who has en una red empresarial(segundo nodo)}
	\label{receptoraswhohasatos2 }	
\end{figure}

Aclaración: En los 2 siguientes gráficos, el nodo apuntado representa a la IP destino mientras que el nodo apuntador es la IP origen.

\begin{figure}[H]
	\centering
	\includegraphics[scale = 0.5]{graficos/emisoras_is_at_red_atos.png}
	\caption{paquetes is at en una empresarial}
	\label{emisorasisatatos1 }	
\end{figure}


\begin{figure}[H]
	\centering
	\includegraphics[scale = 0.3]{graficos/emisoras_is_at_atos.png}
	\caption{paquetes is at en una empresarial(segundo nodo)}
	\label{emisorasisatatos2 }
\end{figure}




\newpage

\section{Discusion}

En esta sección analizaremos los resultados obtenidos.

\subsection{Analisis de entropias}

En primer experimento definimos dos modelos y calculamos las probabilidades de cada IP y la entropía de las fuentes de información
que definimos. En base a estos resultados y con los gráficos de la sección \textbf{envios de paquetes} vamos a intentar realizar análisis
comparando entropías y viendo como impacta el tipo y tamaño de red en las mismas.

Como podemos ver en el cuadro \ref{probabilidadesModel1} hay dos ip que tienen más probabilidad de ser emisoras de un paquete de tipo is at.
Estos dos nodos de la red son los que de alguna manera más impactan en la entropía ya que al ser sus probabilidades valores significativamente más 
altos, incrementan la esperanza de la información.
Además no hay muchos nodos que son emisores de paquete tipo is at, de hecho si vemos la figura \ref{emisorasisat} podemos apreciar que
el tráfico es coherente con las probabilidades.
 
Ahora si miramos las probabilidades de cada Ip de ser receptora de paquetes who has, vamos a notar en el gráfico \ref{histogramaprobabilidadesModel2}
que la IP 192.168.1.1 resalta, probablemente este nodo en la red es el default gateway(y en efecto lo es).
 La entropía en este caso es mucho menor a la anterior, posiblemente porque la distribución de probabilidades es más
pareja, a excepción del dato recién mencionado. 

Otro análisis que se puede hacer en base a las entropías y probabilidades es que en la figura \ref{histogramaprobabilidadesModel1} 
la entropía es mayor a 1 y las probabilidades son más parejas lo cuál significa que la información de un paquete ARP puede revelar 
mucha más información que en el caso de la figura \ref{histogramaprobabilidadesModel2} donde la entropía es menor a 1 y se
destaca la puerta de enlace predeterminada.

Si hacemos este mismo análisis para la red empresaríal vamos a notar que las probabilidaes son mucho más parejas en todos lo casos, pero
si vemos el gráfico \ref{histogramaprobabilidadesatos} vemos que la IP 172.16.189.14 se destaca. Este nodo podría ser el 
más solicitado por ser una puerta de enlace predeterminada, una base de datos o incluso un recurso disponible y muy requerido.

Por otra parte si observamos la tabla \ref{entropiasexperimento2} y lo comparamos con la tabla \ref{entropiashogar} vamos a poder 
observar que para una misma fuente de información las entropías varían drasticamente dependiendo de la topología de la red y su tamaño.
De hecho en el primer modelo tenemos una entropia de 1,37 contra una de 3.53, y en el segundo modelo,una entropía de 0,57 contra una
de 4.352357.

En cuanto a los valores de la entropías con respecto a las probabilidades, en ambos casos es mayor a 1, lo cual nos indica que
en cada emision/recepción de paquete hay información más valiosa.


%Ahora bien podriamos medir que tanto impacta el tamaño según la proporción.
%En la red hogareña hay solo 7 nodos, mientras que en la red empresarial hay aproximadamente 108 nodos.
%Si definimos como índice \textbf{entropia/cantidad de nodos} podemos ver que para la primer fuente cada nodo aporta un valor 0,19, mientras que 
%en la segunda el valor es de tan solo 0.03, es decir la entropía aportada por cada nodo es mucho menor, posiblemente se deba
%a que las probabilidaes sean más parejas.

%Para la segunda fuente ocurre algo similar pero no tan desproporcioando ya que tenemos un valor aproximado de 0.07 en la hogareña contra 
%un 0.04 de la red empresarial y comparado con el caso anterior, la diferencia no es tan grande.

Como conclusiones generales de todo esto podemos ver que:
\begin{itemize}
 \item El tipo de red impacta en las entropías drasticamente.
 \item En el caso del segundo modelo, el impacto que tiene el tamaño de la red es mucho menor. 
 \item En ambos casos podría haber nodos distinguidos, ya sea por emision o recepción de paquetes.
\item En ambas redes hay muchos nodos con probabilidades de emision/recepcion muy parejas, que sin la presencia de otros nodos distinguidos
  incrementarian la entropía de la red. 
\item La red empresarial es bastante heterogenea o bien tiene fragmentos de la misma destinada a proveer servicios.
\end{itemize}


\section{Nodos distinguidos}
Un nodo distinguido es aquel cuya interacción con otros es más frecuente  ya sea como emisor o receptor de paquetes.
Intentaremos encontrarlos en las dos redes estudiadas y además veremos si los resultados encontrados se relacionan con lo mencionado en
la sección anterior.
Al ser la red hogareña una red pequeña, podemos ver en completitud la cantidad de paquetes enviados y recibidos,
no pudimos hacer lo mismo en la red empresarial ya que el gráfico obtenido era ilegible, como antes mencionamos, decidimos truncarlo cortando la parte que creemos
más importante y mostrando solo aquellos nodos que enviaron y recibieron paquetes de una cantidad más alta de la común.

En la figura \ref{emisorasisat} podemos ver los paquetes is-at emitidos en la red hogareña muchos fueron dirigidos a 192.168.1.138 y
existe cierto nivel de interacción entre 192.168.1.143, 192.168.1.1 y 192.168.1.136.

En la figura \ref{emisoraswhohas} podemos ver una cantidad de paquetes mucho mayor pero con una distribución mucho menos centralizada,
esta vez  se distingue el nodo 192.168.1.136.

En el caso de la red empresarial el análisis fue mucho más dificil de ver a simple vista, hubo que aplicar criterios y tomar muestras
por partes. Intentamos lo siguiente:
\begin{itemize}
\item Tomar fragmentos de la red.
\item Reducir la muestra a un tamaño proporcional(teniendo en cuenta la ditribución de probabilidades).
\item Tomar en cuenta solo aquellos nodo que envian o reciben más de cierta cantidad de paquetes.
\end{itemize} 

De estas tres opciones solo la ultima nos dió resultados y luego allí tomamos una muestra.

En las figuras \ref{receptoraswhohasatos} y \ref{receptoraswhohasatos2 } podemos ver una fuerte interacción con el nodo de IP 
172.16.189.86 y un fenómeno mucho más sorprendente es que muy aisladamente el nodo de IP 172.16.189.1 pregunta por 172.16.189.14
y 172.16.189.103 con mucha frecuencia. Logramos determinar que la 172.16.189.14 no se encontraba disponible (ya sea porque no existía 
en la red o porque el dispositivo se encontraba apagado) y por tal motivo la IP 172.16.189.1 envíaba paquetes ARP who-has periodicamente.
Es probable que suceda algo similar con 172.16.189.103.

%Lo malo de haber truncado el grafo de esta manera es que no podemos ver que efectivamente el nodo de IP 172.16.189.14 es el que más
%probabilidad tiene, posiblemente esto sea porque recibe pocos paquetes por parte de cada uno de los otros nodos, pero si 
%sumamos el total, es el que mas recibe. La misma observación no aplica para el nodo de IP 172.16.189.1 donde claramente se condice con 
%lo mostrado en la figura \ref{histogramaprobabilidadesatos}.

En cuanto a la emisión de paquetes is-at podemos contemplar las figuras \ref{emisorasisatatos1 } y \ref{emisorasisatatos2 }, donde pudimos encontrar
dos casos aislados uno de mayor interacción. En uno  muchos respondian al nodo 172.16.189.167 y en otro una cantidad acotada de nodos
enviaba la misma cantidad de paquetes a 172.16.189.190.

De esto podemos sacar las siguientes conclusiones:
\begin{itemize}
\item Las dos redes presentan nodos aislados a pesar de su distinta utilidad y tamaño, posiblemente se deba a los protocolos de comunicación y armado de tablas de ruteo. 
\item En el caso de la red empresarial hay una relación entre los gráficos de probabilidades de receptores de paquetes who-has y nodos aislados. 
\item La red empresarial presenta muchas más interacciones aisladas, posiblemente tenga una cantidad mayor de dispositivos presten servicios muy específicos.

\end{itemize}

 



 

\newpage

\subsection{Conclusiones}

Para cerrar este informe queremos plantear nuestras conclusiones provenientes de la expermientación de modelar una fuente de información usando datos de capturas de paquetes de red, analizando la frecuencia con la cual aparecía cada protocolo o dirección IP. Observamos que existe una relación entre frecuencia de aparición y cantidad de información aportada por el evento que consiste en ver ese paquete en la red. Esto se debe a que nostros calculamos la probabilidad de que suceda un evento en base a las frecuencias relativas de aparición de los paquetes en la red, y como el cálculo de información aportada por un evento considera la inversa de esta probabilidad crea una relación inversamente proporcional entre ambas métricas.

Primero analizando la aparición de paquetes por protocolo podemos apreciar que mucho del tráfico generado en una red se debe principalmente al intercambio de paquetes IPv4 entre nodos de la red o hacia Internet, mientras que el protocolo ARP sólo representaba una pequeña porción. Sin embargo, esos paquetes ARP son los que más información aportan sobre la red y el cálculo de información aportada está de acuerdo con ello. Luego podemos ver cómo cambia la entropía dependiendo del tráfico en la red. Si hay muchos paquetes del mismo tipo, esto implica que la probabilidad el próximo paquete que aparece en la red sea de ese tipo va aumentando y por lo tanto sabemos más sobre la red. Esto se ve reflejado en la entropía, porque cuando un paquete era más frecuente la entropía disminuía porque la incertidumbre del tipo del próximo paquete era más baja. Adicionalmente podemos ver la poca presencia que tienen los paquetes IPv6 en las redes hoy en día, pero es cuestión de tiempo hasta que esto se vaya revirtiendo ya que IPv6 puede reemplazar a IPv4 por la creciente demanda de direcciones IP a nivel mundial.

Segundo analizamos la topología de la redes también modelando el tráfico como una fuente de información que emite direcciones IP provenientes de paquetes ARP de la red. Con esta técnica podemos apreciar qué nodos sobresalen del resto, pero a veces por cuestiones distintas. En la red de la universidad podíamos ver, analizando el tráfico, que había mucho reenvío de paquetes y por lo tanto las probabilidades de que aparezca una dirección IP aumentaban en cuanto a la frecuencia relativa. En este caso también podía deberse a que la red estaba sobrecargada y si bien los nodos probablemente respondían con un paquete IS\_AT, éste nunca llegaba. En redes más controladas, pudimos ver qué un nodo principal correspondía con la dirección que generalmente tienen los routers. Como era una red de acceso público en una cafetería tiene sentido que este sea el nodo distinguido ya que dificilmente dos dispositivos en una red pública quieran comunicarse entre sí, ya que el uso principal que se les da es para acceder a Internet. Además en una red pública es esperable que nuevos dispositivos se conecten a la red en cualquier momento, entonces siempre van a necesitar saber la dirección física del router mediante el uso de paquetes ARP. Finalmente en las redes de ambientes laborales podemos ver que no es el router el nodo más distinguido ya que es posible que haya otro recursos compartidos dentro de la red del trabajo que son más requeridos que el acceso a Internet.	
\newpage

%\input{apendices.tex}	
%\newpage

%\input{referencias.tex}

\end{document}