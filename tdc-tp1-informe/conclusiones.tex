\subsection{Conclusiones}

Para cerrar este informe queremos plantear nuestras conclusiones provenientes de la expermientación de modelar una fuente de información usando datos de capturas de paquetes de red, analizando la frecuencia con la cual aparecía cada protocolo o dirección IP. Observamos que existe una relación entre frecuencia de aparición y cantidad de información aportada por el evento que consiste en ver ese paquete en la red. Esto se debe a que nostros calculamos la probabilidad de que suceda un evento en base a las frecuencias relativas de aparición de los paquetes en la red, y como el cálculo de información aportada por un evento considera la inversa de esta probabilidad crea una relación inversamente proporcional entre ambas métricas.

Primero analizando la aparición de paquetes por protocolo podemos apreciar que mucho del tráfico generado en una red se debe principalmente al intercambio de paquetes IPv4 entre nodos de la red o hacia Internet, mientras que el protocolo ARP sólo representaba una pequeña porción. Sin embargo, esos paquetes ARP son los que más información aportan sobre la red y el cálculo de información aportada está de acuerdo con ello. Luego podemos ver cómo cambia la entropía dependiendo del tráfico en la red. Si hay muchos paquetes del mismo tipo, esto implica que la probabilidad el próximo paquete que aparece en la red sea de ese tipo va aumentando y por lo tanto sabemos más sobre la red. Esto se ve reflejado en la entropía, porque cuando un paquete era más frecuente la entropía disminuía porque la incertidumbre del tipo del próximo paquete era más baja. Adicionalmente podemos ver la poca presencia que tienen los paquetes IPv6 en las redes hoy en día, pero es cuestión de tiempo hasta que esto se vaya revirtiendo ya que IPv6 puede reemplazar a IPv4 por la creciente demanda de direcciones IP a nivel mundial.

Segundo analizamos la topología de la redes también modelando el tráfico como una fuente de información que emite direcciones IP provenientes de paquetes ARP de la red. Con esta técnica podemos apreciar qué nodos sobresalen del resto, pero a veces por cuestiones distintas. En la red de la universidad podíamos ver, analizando el tráfico, que había mucho reenvío de paquetes y por lo tanto las probabilidades de que aparezca una dirección IP aumentaban en cuanto a la frecuencia relativa. En este caso también podía deberse a que la red estaba sobrecargada y si bien los nodos probablemente respondían con un paquete IS\_AT, éste nunca llegaba. En redes más controladas, pudimos ver qué un nodo principal correspondía con la dirección que generalmente tienen los routers. Como era una red de acceso público en una cafetería tiene sentido que este sea el nodo distinguido ya que dificilmente dos dispositivos en una red pública quieran comunicarse entre sí, ya que el uso principal que se les da es para acceder a Internet. Además en una red pública es esperable que nuevos dispositivos se conecten a la red en cualquier momento, entonces siempre van a necesitar saber la dirección física del router mediante el uso de paquetes ARP. Finalmente en las redes de ambientes laborales podemos ver que no es el router el nodo más distinguido ya que es posible que haya otro recursos compartidos dentro de la red del trabajo que son más requeridos que el acceso a Internet.