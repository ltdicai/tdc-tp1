\section{Experimentos}

Como mencionamos anteriormente, el análisis de paquetes de una red puede utilizarse para inferir información sobre la actividad y topología de la red. En este trabajo aprovecharemos esta capacidad para dilucidar qué protocolos se distinguen del resto, cuál es la incidencia de los paquetes ARP y cuáles son los nodos destacados de las redes.
Realizamos cuatro experimentos para obtener datos, uno sobre cada una de las siguientes redes:
\begin{itemize}
	\item Red1: Red WiFi de un laboratorio del DC
	\item Red2: Red Wifi de un bar Starbucks
	\item Red3: Red Ethernet en un ámbito laboral
	\item Red4: Red Ethernet en un ámbito laboral
\end{itemize}

Modelamos estas redes como dos fuentes de información distintas:
\begin{enumerate}
	\item $S$: este modelo fue dado por la cátedra. El alfabeto se define como los protocolos enviados dentro de los paquetes Ethernet capturados durante el experimento. Así mismo, consideramos como función de probabilidad a la frecuencia de cada símbolo dentro del experimento, donde marcamos como ocurrencia de un evento a la observación de un protocolo al capturar un paquete.
	\item $S1$: con este modelo deseamos poder distinguir los nodos relevantes de una red data. Para ello definimos el alfabeto de $S1$ como las direcciones IP destino de los paquetes WHO\_HAS del protocolo ARP.
	
La decisión de utilizar las direcciones IP es porque nos interesa saber cuál es la dirección IP del nodo que es más requerido en la red, lo que implicaría que es un nodo central a la red. Podría ser algún router, un servidor o una impresora de red, por nombrar algunos.

\end{enumerate}

\subsection{Herramientas de \emph{sniffing}}

Para capturar y procesar la información, utilizamos tanto el programa \textit{Wireshark} como dos script (\texttt{capturar.py}, \texttt{identificar.py}), escritos en Python, utilizando la librería para análisis de redes \texttt{scapy}. Ambas herramientas hacen uso del modo promiscuo de la placa de red, en el cual se capturan no solo los paquetes dirigidos a el host que esta capturando, sino todos los paquetes que se envíen por el medio.

\subsubsection{Implementación de $S$: \texttt{capturar.py}}
En su forma de ejecución básica, el script muestra por pantalla cada paquete que captura hasta que sea detenido con una interrupción (\texttt{CTRL+C}). Al finalizar, se muestra 
\begin{enumerate}
	\item el total de paquetes capturados
	\item los protocolos observados (junto con la cantidad de veces que se observó cada uno)
	\item la entropía correspondiente modelo $S$. 
\end{enumerate}
Si bien incorporamos varias opciones de ejecución (ejecutar el comando con la opción \texttt{-h}.), la forma más sencilla corresponde a 
\begin{verbatim}
sudo python capturar.py -i <interfaz_de_captura>
\end{verbatim}

\subsubsection{Implementación de $S1$: \texttt{identificar.py}}
Este script es similar a \texttt{capturar.py}, pero en lugar de analizar los protocolos de cada paquete, filtra sólo los paquetes ARP e implementa el modelo de fuente $S1$. Al igual que \texttt{capturar.py}, el script muestra por pantalla cada paquete que captura hasta que sea detenido con una interrupción (\texttt{CTRL+C}). Al finalizar, devuelve
\begin{enumerate}
	\item un diccionario donde se mapea direcciones MAC con direcciones IP
	\item un diccionario de direcciones IP observadas, y su cantidad de veces que fueron observadas
	\item las direcciones MAC observadas (junto con la cantidad de veces que se observo cada una)
	\item la entropía correspondiente modelo $S1$. 
\end{enumerate}
Para ver opciones de ejecución, ejecutar el comando con \texttt{-h}.). la forma más sencilla corresponde a 
\begin{verbatim}
sudo python identificar.py -i <interfaz_de_captura>
\end{verbatim}
