\section{Desarrollo}
En este inciso explicaremos como se realizó cada ejercicio del trabajo práctico.

\subsection{Herramienta para escuchar}
Decidimos usar scapy por simplicidad y lo desarrollado en los talleres para escuchar pasivamente la red local

El código de la herrmienta es el siguiente:

\begin{center}
 \begin{verbatim}
  #! /usr/bin/python
  from scapy.all import *

  def monitor_callback(pkt):
    print pkt.show()

  sniff(prn=monitor_callback,filter="arp",store=0)
 \end{verbatim}

\end{center}

En este caso hacemos uso de la función sniff para ver todos los paquetes tipo ARP que se envían.


\subsection{Análisis de entropía de una red}
Uno de los ejercicios solicitaba calcular la entropía de una fuente.
Para ello debemos definir con presicion dos cosas, la \textbf{fuente de informacion} y el \textbf{evento}, para luego calcular su probabilidad
y de allí la entropía.

\subsubsection{Experimento 1:}

Definimos los siguientes modelos de fuente de información en un red. Previamente contamos con información de cómo está compuesta dicha red: 2 Notebooks, 2 Smartphones, 1 SmartTV, 1 Desktop/Media Server y 1 Router.

Calcularemos las entropías modificando la \textit{tool} construida. 

\textbf{Modelo 1:}
\begin{itemize}
\item \textbf{Fuente de información:} Considero un símbolo como una dirección IP que emite un paquete ARP del tipo \textbf{is-at}.
\item \textbf{Probabilidad de evento:} Probabilidad de que cierta IP aparezca como emisora de un paquete \textbf{is-at}.
\end{itemize}

\newpage
\textbf{pseudocódigo}:
\begin{algorithm}
\begin{algorithmic}
\REQUIRE true
\ENSURE entropia de una fuente.
\STATE ipssrc = Diccionario(IP,int).
\IF{ARP in pkt AND pkt.op = is-at}
\STATE src = pkt[ARP].psrc
\STATE ipssrc[src] = ipssrc[src]+1.
\ENDIF
\RETURN entropia(ipssrc)
\end{algorithmic}
\caption{callback(pkt)}
\end{algorithm}


\textbf{Modelo 2:}
\begin{itemize}
\item \textbf{Fuente de información:} Considero un símbolo como una dirección IP por la que se consulta en un paquete ARP del tipo \textbf{who-has}.
\item \textbf{Probabilidad de evento:} Probabilidad de que cierta IP aparezca como destino de un paquete \textbf{who-has}.
\end{itemize}

\textbf{pseudocódigo}:
\begin{algorithm}
\begin{algorithmic}
\REQUIRE true
\ENSURE entropia de una fuente.
\STATE ipsdst = Diccionario(IP,int).
\IF{ARP in pkt AND pkt.op = who-has}
\STATE dst = pkt[ARP].pdst
\STATE ipsdst[src] = ipsdst[dst]+1.
\ENDIF
\RETURN entropia(ipsdst)
\end{algorithmic}
\caption{callback(pkt)}
\end{algorithm}

\subsubsection{Experimento 2:}
En nuestro segundo experimento, aplicaremos los modelos definidos en el \textit{Experimento 1} a una red empresarial
con el fin de analizar entropías y al ser una red más grande buscar nodos distinguidos como podrían ser servidores de archivos, 
base de datos, servidores de email e inclusive la \textit{puerta de enlace predeterminada}.

%~ \textbf{Modelo 3:}
%~ \begin{itemize}
 %~ \item Fuente de información: Considero un símbolo como una dirección IP que recibe un paquete ARP del tipo \textbf{is-at}.
%~ \item Probabilidad de evento: Probabilidad que cierta IP aparezca  como receptora de un paquete \textbf{is-at}.
%~ \end{itemize}
%~ 
%~ \textbf{pseudocódigo tool utilizada}:
%~ \begin{algorithm}
%~ \begin{algorithmic}
%~ \REQUIRE true
%~ \ENSURE entropia de una fuente.
%~ \STATE ipsrcv = Diccionario(IP,int).
%~ \IF{ARP in pkt AND pkt.op =isat}
%~ \STATE dst= pkt[ARP].pdst
%~ \STATE ipsdst[src] = ipsdst[dst]+1.
%~ \ENDIF
%~ \RETURN entropia(ipsdst)
%~ \end{algorithmic}
%~ \caption{callback(pkt)}
%~ \end{algorithm}
%~ 
%~ 
%~ \textbf{Modelo 4:}
%~ \begin{itemize}
 %~ \item Fuente de información: Considero un símbolo como una dirección IP que recibe un paquete ARP del tipo \textbf{who-has}.
%~ \item Probabilidad de evento: Probabilidad que cierta IP aparezca  como receptora de un paquete \textbf{who-has}.
%~ \end{itemize}
%~ 
%~ \textbf{pseudocódigo tool utilizada}:
%~ \begin{algorithm}
%~ \begin{algorithmic}
%~ \REQUIRE true
%~ \ENSURE entropia de una fuente.
%~ \STATE ipsrcv = Diccionario(IP,int).
%~ \IF{ARP in pkt AND pkt.op =who-has}
%~ \STATE dst= pkt[ARP].pdst
%~ \STATE ipsdst[src] = ipsdst[dst]+1.
%~ \ENDIF
%~ \RETURN entropia(ipsdst)
%~ \end{algorithmic}
%~ \caption{callback(pkt)}
%~ \end{algorithm}






%Teniendo en cuanta esto ahora podemos aplicar la formula \ref{entropia } porque podemos definir un tiempo limitado hasta cuanto 
%capturar paquetes y luego contar la frecuencia de cada simbolo para definir su probabilidad.



