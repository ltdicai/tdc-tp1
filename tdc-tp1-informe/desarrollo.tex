\section{Experimentos}

Como mencionamos anteriormente, el análisis de paquetes de una red puede utilizarse para inferir información sobre la actividad y topología de la red. En este trabajo, aprovecharemos esta capacidad para encontrar qué protocolos se distinguen del resto, cuál es la incidencia de los paquetes ARP y cuáles son los nodos destacados.
Realizamos cuatro experimentos para obtener datos. Cada uno se realizó sobre una de las siguientes redes:
\begin{itemize}
	\item Red1: Red wiFi de un laboratorio del DC
	\item Red2: Red Wifi de un bar Starbucks
	\item Red3: Red Ethernet de un ámbito laboral
	\item Red4: Red Ethernet de un ámbito laboral
\end{itemize}

Modelamos estas redes como dos fuentes de información distintas:

\begin{itemize}
	\item $S$: el alfabeto se define como los protocolos que se pueden enviar sobre los paquetes Ethernet/WiFi. Tomamos la probabilidad de cada uno como la frecuencia dentro de cada captura.
	\item $S1$: definimos el alfabeto como las direcciones MAC de los paquetes del protocolo ARP. Consideramos  que se pueden enviar sobre los paquetes Ethernet/WiFi. Tomamos la probabilidad de cada uno como la frecuencia dentro de cada captura.

\end{itemize}


\subsection{Herramientas de sniffing}

Para capturar y procesar la información, utilizamos tanto el programa \textit{Wireshark} como un script (capturar.py), escrito en Python, utilizando la librería para análisis de redes \textit{scapy}. Ambas herramientas hacen uso del modo promíscuo de la placa de red, en el cual se capturan no solo los paquetes dirigidos a el host que esta capturando, sino todos los paquetes que se envíen por el medio.

En su forma de ejecución básica, el script que desarrollamos muestra por pantalla cada paquete que captura hasta que sea detenido con una interrupción (CTRl+C). Al finalizar, se muestra el total de paquetes capturados, los protocolos de dichos paquetes (junto con la cantidad correspondiente), y la entropía del mode

 la cantidad de paquetes  Incorporamos otros modos de ejecución.se realiza a través del comando siguiente comando (para ver otras opciónes , ejecutar el comando con la opción \textit{-h}.)

\begin{verbatim}
	sudo python capturar.py -i <interfaz_de_captura>
\end{verbatim}



\subsection{Análisis de entropía de una red}
Uno de los ejercicios solicitaba calcular la entropía de una fuente.
Para ello debemos definir con presicion dos cosas, la \textbf{fuente de informacion} y el \textbf{evento}, para luego calcular su probabilidad
y de allí la entropía.

\subsubsection{Experimento 1:}

Definimos los siguientes modelos de fuente de información en un red. Previamente contamos con información de cómo está compuesta dicha red: 2 Notebooks, 2 Smartphones, 1 SmartTV, 1 Desktop/Media Server y 1 Router.

Calcularemos las entropías modificando la \textit{tool} construida. 

\textbf{Modelo 1:}
\begin{itemize}
\item \textbf{Fuente de información:} Considero un símbolo como una dirección IP que emite un paquete ARP del tipo \textbf{is-at}.
\item \textbf{Probabilidad de evento:} Probabilidad de que cierta IP aparezca como emisora de un paquete \textbf{is-at}.
\end{itemize}

\newpage
\textbf{pseudocódigo}:
\begin{algorithm}
\begin{algorithmic}
\REQUIRE true
\ENSURE entropia de una fuente.
\STATE ipssrc = Diccionario(IP,int).
\IF{ARP in pkt AND pkt.op = is-at}
\STATE src = pkt[ARP].psrc
\STATE ipssrc[src] = ipssrc[src]+1.
\ENDIF
\RETURN entropia(ipssrc)
\end{algorithmic}
\caption{callback(pkt)}
\end{algorithm}


\textbf{Modelo 2:}
\begin{itemize}
\item \textbf{Fuente de información:} Considero un símbolo como una dirección IP por la que se consulta en un paquete ARP del tipo \textbf{who-has}.
\item \textbf{Probabilidad de evento:} Probabilidad de que cierta IP aparezca como destino de un paquete \textbf{who-has}.
\end{itemize}

\textbf{pseudocódigo}:
\begin{algorithm}
\begin{algorithmic}
\REQUIRE true
\ENSURE entropia de una fuente.
\STATE ipsdst = Diccionario(IP,int).
\IF{ARP in pkt AND pkt.op = who-has}
\STATE dst = pkt[ARP].pdst
\STATE ipsdst[src] = ipsdst[dst]+1.
\END