\section{Introducción}

Las redes de computadoras han dejado de ser 
una tecnología reservada a ciertos ámbitos científicos y militares para convertirse
en piezas fundamentales en el desarrollo de casi cualquier actividad, a tal grado que las relaciones humanas, 
desde el comercio hasta las guerras, han sido profundamente transformadas por la conectividad
alcanzada en los últimos años.  
Es por esto que analizar los distintos aspectos de una red
puede proveer información útil para comprender el uso que se le está dando a la red, información que sirve tanto para
modificar la infraestructura y los protocolos
utilizados a fin de mejorar la calidad del servicio como, incluso, manipular las actividades que se esten llevando sobre ella.


En el presente trabajo experimentaremos sobre sistemas basados en dos de las tecnologías de redes más difundidas, Ethernet (802.3) y WiFi (802.11), y analizaremos los datos obtenidos utilizando dos modelos de fuente de información para extraer conclusiones sobre el uso y la configuración de las redes. Los conceptos teóricos sobre los que basaremos el análisis se presentan a continuación.


\subsection{Información y Fuente de información}

Una fuente de información es todo aquello que emite mensajes de acuerdo a una ley de probabilidad fija. Los mensajes pertenecen a un conjunto finito de símbolos $S={s_{1},...,s_{n}}$, conocido como el alfabeto de la fuente. La emisión de un símbolo $s_i$ por parte de la fuente $S$ representa un evento que tiene asociada una probabilidad fija $P_S(s_i)$ de ocurrir. 

Dado un evento $e$ con probabilidad $P(e)$, se define la \textbf{información del evento $e$} como

\begin{center}
$I(e)=-\log{P(e)}$ 
\end{center}

$I(e)$ es una medida de la cantidad de información que obtenemos por la ocurrencia de E: mientras más improbable sea E, mayor será la información brindada por su ocurrencia (menor será la incertidumbre sobre el hecho observado). Dicho de otra manera, si sabemos que un evento E tiene alta probabilidad de ocurrir, entonces su ocurrencia no aportará mucha información sobre lo que se está observando.



\subsection{Entropía}

Dada una fuente de información $S={s_{1},...,s_{n}}$, se define la entropía de S, $H(S)$, como la suma ponderada de la información de cada símbolo de $S$

$$H(S)=\sum^{n}_{i=1}{P(s_{i})*I(s_{i})}$$

La entropía de una fuente de información mide la cantidad de información esperada al observar la emisión de un nuevo símbolo por parte de la fuente. 

\subsection{ARP}

Para poder realizar envío de paquetes de capa de red utilizando los servicios de capa de enlace es necesario poder realizar un mapeo entre las direcciones de ambas capas. ARP (\textsl{Address Resolution Protocol}) es un protocolo de control que surge como respuesta a esta necesidad. Cada host y switch de una red mantiene una tabla donde se relaciona una dirección lógica $d$ con la direccion física $f$ a la que debe entregarse cualquier paquete destinado a $d$ (el host con direccion física $f$ no es necesariamente el destinatario de la dirección $d$: puede ser un intermediario que sabe cómo hacer llegar el paquete a $d$). En el caso de redes IP sobre Ethernet, el protocolo ARP es utilizado para mapear direcciones IP con direcciones MAC.
La configuración de estas tablas ARP se realiza dinámicamente siguiendo un protocolo que consiste básicamente en los siguientes pasos:

\begin{enumerate}
	\item Un host \textbf{A} desea enviar un paquete a una determinada IP. Si \textbf{A} conoce la dirección MAC a la que debe enviar los paquetes destinados a esa IP, entonces utiliza esa dirección física. Si no, envía un mensaje $broadcast$, o sea a todos los hosts dentro de la red, y aguarda la respuesta. Este mensaje se conoce como \textbf{ARP request} (WHO\_HAS), y lleva la siguiente información:
		\subitem IP origen: IP de A
		\subitem IP destino: IP a la que se desea enviar un paquete
		\subitem MAC origen: MAC de A
		\subitem MAC destino: dirección broadcast de Ethernet (FF:FF:FF:FF)
	\item Si dentro de la red existe un host \textbf{B} que sabe cómo direccionar a la dirección IP requerida, entonces responde al mensaje ARP request con un mensaje \textbf{ARP reply} (IS\_AT) indicando su dirección física MAC. Este host puede ser el dueño de la dirección IP, o un host intermediario (como un router). Además, extrae las direcciones IP origen y MAC origen del paquete ARP request, y actualiza su tabla ARP para relacionarlas.
	El paquete ARP reply contiene la siguiente información:
		\subitem IP origen: IP de B
		\subitem IP destino: IP de A
		\subitem MAC origen: MAC de B
		\subitem MAC destino: MAC de A
	\item \textbf{A} recibe el ARP reply de \textbf{B}, actualiza su tabla ARP y envía el paquete original utilizando la dirección física de \textbf{B}.
\end{enumerate}
Además, cada entrada de las tablas ARP tiene seteado un tiempo de vida. Una vez agotado ese tiempo, la entrada se descarta y debe volver a aprenderse. 


