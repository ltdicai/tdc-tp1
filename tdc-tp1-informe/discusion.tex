\section{Discusión}

En esta sección analizaremos los resultados obtenidos de los experimentos acerca de paquetes de una red interpretados como una fuente de información.

\subsection{Análisis de entropias}

En primer experimento definimos dos modelos y calculamos las probabilidades de cada IP y la entropía de las fuentes de información
que definimos. En base a estos resultados y con los gráficos de la sección \textbf{envios de paquetes} vamos a realizar análisis
comparando entropías y viendo como impacta el tipo y tamaño de red en las mismas.

Como podemos ver en el cuadro \ref{probabilidadesModel1} hay dos ip que tienen más probabilidad de ser emisoras de un paquete de tipo is at.
Estos dos nodos de la red son los que de alguna manera más impactan en la entropía ya que al ser sus probabilidades valores significativamente más 
altos, incrementan la esperanza de la información.
Además no hay muchos nodos que son emisores de paquete tipo is at, de hecho si vemos la figura \ref{emisorasisat} podemos apreciar que
el tráfico es coherente con las probabilidades.
 
Ahora si miramos las probabilidades de cada Ip de ser receptora de paquetes who has, vamos a notar en el gráfico \ref{histogramaprobabilidadesModel2}
que la IP 192.168.1.1 resalta, probablemente este nodo en la red es el default gateway(y en efecto lo es).
 La entropía en este caso es mucho menor a la anterior, posiblemente porque la distribución de probabilidades es más
pareja, a excepción del dato recién mencionado. 

Otro análisis que se puede hacer en base a las entropías y probabilidades es que en la figura \ref{histogramaprobabilidadesModel1} 
la entropía es mayor a 1 y las probabilidades son más parejas lo cuál significa que la información de un paquete ARP puede revelar 
mucha más información que en el caso de la figura \ref{histogramaprobabilidadesModel2} donde la entropía es menor a 1 y se
destaca la puerta de enlace predeterminada.

Si hacemos este mismo análisis para la red empresaríal vamos a notar que las probabilidaes son mucho más parejas en todos lo casos, pero
si vemos el gráfico \ref{histogramaprobabilidadesatos} vemos que la IP 172.16.189.14 se destaca. Este nodo podría ser el 
más solicitado por ser una puerta de enlace predeterminada, una base de datos o incluso un recurso disponible y muy requerido.

Por otra parte si observamos la tabla \ref{entropiasexperimento2} y lo comparamos con la tabla \ref{entropiashogar} vamos a poder 
observar que para una misma fuente de información las entropías varían drasticamente dependiendo de la topología de la red y su tamaño.
De hecho en el primer modelo tenemos una entropia de 1,37 contra una de 3.53, y en el segundo modelo,una entropía de 0,57 contra una
de 4.352357.

En cuanto a los valores de la entropías con respecto a las probabilidades, en ambos casos es mayor a 1, lo cual nos indica que
en cada emision/recepción de paquete hay información más valiosa.


%Ahora bien podriamos medir que tanto impacta el tamaño según la proporción.
%En la red hogareña hay solo 7 nodos, mientras que en la red empresarial hay aproximadamente 108 nodos.
%Si definimos como índice \textbf{entropia/cantidad de nodos} podemos ver que para la primer fuente cada nodo aporta un valor 0,19, mientras que 
%en la segunda el valor es de tan solo 0.03, es decir la entropía aportada por cada nodo es mucho menor, posiblemente se deba
%a que las probabilidaes sean más parejas.

%Para la segunda fuente ocurre algo similar pero no tan desproporcioando ya que tenemos un valor aproximado de 0.07 en la hogareña contra 
%un 0.04 de la red empresarial y comparado con el caso anterior, la diferencia no es tan grande.

Como conclusiones generales de todo esto podemos ver que:
\begin{itemize}
 \item El tipo de red impacta en las entropías drasticamente.
 \item En el caso del segundo modelo, el impacto que tiene el tamaño de la red es mucho menor. 
 \item En ambos casos podría haber nodos distinguidos, ya sea por emision o recepción de paquetes.
\item En ambas redes hay muchos nodos con probabilidades de emision/recepcion muy parejas, que sin la presencia de otros nodos distinguidos
  incrementarian la entropía de la red. 
\item La red empresarial es bastante heterogenea o bien tiene fragmentos de la misma destinada a proveer servicios.
\end{itemize}


\section{Nodos distinguidos}
Un nodo distinguido es aquel cuya interacción con otros es más frecuente  ya sea como emisor o receptor de paquetes.
Intentaremos encontrarlos en las dos redes estudiadas y además veremos si los resultados encontrados se relacionan con lo mencionado en
la sección anterior.
Al ser la red hogareña una red pequeña, podemos ver en completitud la cantidad de paquetes enviados y recibidos,
no pudimos hacer lo mismo en la red empresarial ya que el gráfico obtenido era ilegible, como antes mencionamos, decidimos truncarlo cortando la parte que creemos
más importante y mostrando solo aquellos nodos que enviaron y recibieron paquetes de una cantidad más alta de la común.

En la figura \ref{emisorasisat} podemos ver los paquetes is-at emitidos en la red hogareña muchos fueron dirigidos a 192.168.1.138 y
existe cierto nivel de interacción entre 192.168.1.143, 192.168.1.1 y 192.168.1.136.

En la figura \ref{emisoraswhohas} podemos ver una cantidad de paquetes mucho mayor pero con una distribución mucho menos centralizada,
esta vez  se distingue el nodo 192.168.1.136.

En el caso de la red empresarial el análisis fue mucho más dificil de ver a simple vista, hubo que aplicar criterios y tomar muestras
por partes. Intentamos lo siguiente:
\begin{itemize}
\item Tomar fragmentos de la red.
\item Reducir la muestra a un tamaño proporcional(teniendo en cuenta la ditribución de probabilidades).
\item Tomar en cuenta solo aquellos nodo que envian o reciben más de cierta cantidad de paquetes.
\end{itemize} 

De estas tres opciones solo la ultima nos dió resultados y luego allí tomamos una muestra.

En las figuras \ref{receptoraswhohasatos} y \ref{receptoraswhohasatos2 } podemos ver una fuerte interacción con el nodo de IP 
172.16.189.86 y un fenómeno mucho más sorprendente es que muy aisladamente el nodo de IP 172.16.189.1 pregunta por 172.16.189.14
y 172.16.189.103 con mucha frecuencia. Logramos determinar que la 172.16.189.14 no se encontraba disponible (ya sea porque no existía 
en la red o porque el dispositivo se encontraba apagado) y por tal motivo la IP 172.16.189.1 envíaba paquetes ARP who-has periodicamente.
Es probable que suceda algo similar con 172.16.189.103.

%Lo malo de haber truncado el grafo de esta manera es que no podemos ver que efectivamente el nodo de IP 172.16.189.14 es el que más
%probabilidad tiene, posiblemente esto sea porque recibe pocos paquetes por parte de cada uno de los otros nodos, pero si 
%sumamos el total, es el que mas recibe. La misma observación no aplica para el nodo de IP 172.16.189.1 donde claramente se condice con 
%lo mostrado en la figura \ref{histogramaprobabilidadesatos}.

En cuanto a la emisión de paquetes is-at podemos contemplar las figuras \ref{emisorasisatatos1 } y \ref{emisorasisatatos2 }, donde pudimos encontrar
dos casos aislados uno de mayor interacción. En uno  muchos respondian al nodo 172.16.189.167 y en otro una cantidad acotada de nodos
enviaba la misma cantidad de paquetes a 172.16.189.190.

De esto podemos sacar las siguientes conclusiones:
\begin{itemize}
\item Las dos redes presentan nodos aislados a pesar de su distinta utilidad y tamaño, posiblemente se deba a los protocolos de comunicación y armado de tablas de ruteo. 
\item En el caso de la red empresarial hay una relación entre los gráficos de probabilidades de receptores de paquetes who-has y nodos aislados. 
\item La red empresarial presenta muchas más interacciones aisladas, posiblemente tenga una cantidad mayor de dispositivos presten servicios muy específicos.

\end{itemize}

 



 
